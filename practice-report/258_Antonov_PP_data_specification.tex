\section{Спецификация данных}
\subsection{Формат метафайла}
Метафайл -- файл, используемый для сохранения и последующего восстановления информации о загруженных пользователем изображениях, введенных матрицах внешней и внутренней калибровки, объемлющем основной объект прямоугольнике. Этот файл записан в текстовом формате. Файл имеет следующую структуру:
\begin{itemize}
\item В первой строке находится беззнаковое целое число, обозначающее количество изображений, информация о которых записана в файле.
\item Далее идет три строки, содержащие по три элемента, разделенных пробелами. Каждый элемент -- число с плавающей точкой. Вместе они описывают матрицу внутренней калибровки камеры.
\item Далее идет информация об изображениях:
	\begin{itemize}
		\item имя файла с изображением, записанное в формате, совместимом с типом QString (см.~\cite{qt_qstring})
		\item прямоугольник, объемлющий основной объект, записанный в формате, совместимом с типом QRectF (см.~\cite{qt_qrectf})
		\item матрица внешней калибровки камеры для данного изображения, записанная в три строки по три числа с плавающей точкой в каждой.
	\end{itemize}
Количество таких блоков должно быть равным числу, записанном в первой строке файла.
\end{itemize}

