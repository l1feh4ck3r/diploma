\section{Введение}

\subsection{Глоссарий}
\textbf{Воксел} -- элемент объёмного изображения, содержащий значение элемента растра в трёхмерном пространстве. Вокселы являются аналогами пикселов для трехмёрного пространства.~\cite{wiki_voxel}

\textbf{Кроссплатформенное программное обеспечение} -- программное обеспечение, работающее более чем на одной аппаратной платформе и/или операционной системе.~\cite{wiki_crossplatfom}

\textbf{ПК} -- персональный компьютер.

\textbf{Реконструкция} -- построение объемной модели реального объекта.

\textbf{Фреймворк} -- в информационных системах структура программной системы; программное обеспечение, облегчающее разработку и объединение разных компонентов большого программного проекта. В его состав могут входить вспомогательные программы, библиотеки кода, язык сценариев и проч.~\cite{wiki_framework}

\subsection{Описание предметной области}
Потребность в реалистичном отображении окружающего мира увеличивает значимость трехмерного (3D) моделирования -- построения компьютерных моделей объектов. 3D модели облегчают планирование, контроль и принятие решений во многих отраслях. Например, если в ходе эксплуатации модели требуется внести коррективы, компьютерный объект позволит это сделать максимально быстро.

Особенно сложной проблемой является создание точных моделей объектов из реального мира.~\cite{komarova_voxel_coloring} Как быстро выяснилось, человеческий глаз очень легко определяет погрешности в синтезированных изображениях хорошо известных ему в повседневной жизни объектов.

Вот несколько примеров объектов, требующих реконструкции:
\begin{enumerate}
\item Здания и строения
\item Предприятия со сложной структурой (нефтегазоперерабатывающие комплексы, химические предприятия и т.д.)
\item Дороги и дорожные объекты (мосты, путепроводы, прилегающая зона)
\item Открытые и закрытые горные разработки
\item Рельефы местности
\end{enumerate}

\subsubsection{Методы построения моделей объектов}
На данный момент существует три основных способа получения модели объекта:
\begin{itemize}
\item Использование активных методов (например, лазерных сканеров, позволяющих с высокой точностью определить расстояние от лазера до поверхности объекта)
\item Вручную с помощью программ трехмерного моделирования (например, Blender, 3D Studio Max)
\item Автоматическая реконструкция объекта по набору изображений и другой известной о нем информации
\end{itemize}

Сканеры -- современное средство измерения, позволяющее производить исполнительную съемку сложных объектов, там, где проходит множество пересечений на различных уровнях. Причем вступать в контакт с самим объектом не требуется. Сканер формирует модель по десяткам тысяч точек, при этом положение каждой из них определяется с точностью от 2--3 мм до 2 см. Наземные сканеры используются для создания моделей зданий, заводов, кварталов и других больших сооружений. Дальность сканера составляет порядка 200 метров, на таком расстоянии он может определять объекты с точностью шага до 0.5 см. Современные сканеры в состоянии сканировать окружающую действительность с поворотом на 360 градусов, что позволяет сканировать сразу несколько зданий. Сшивка отдельных сканов выполняется посредством специальных марок или характерных точек рельефа. Такой метод обеспечивает высокое качество проектирования, приближая его к реальности. Недостатками использования сканера являются высокая стоимость оборудования (от 50 тыс. долл.~\cite{laser_scanner}), необходимость в высококвалифицированных специалистах, способных это оборудование обслуживать, а так же невозможность реконструкции динамических сцен.~\cite{komarova_voxel_coloring}

Второй подход требует больших затрат усилий и времени, что вытекает в высокую стоимость и длительные сроки разработки модели.

В третьем подходе в качестве исходных данных используют только фотографии или последовательности изображений объектов, полученные в естественном освещении, или приближенном к нему (в свете обычных ламп). Такой подход считается наиболее перспективными, так как позволяет строить модели динамических сцен, в которых могут участвовать люди.

\subsection{Неформальная постановка задачи}
Создание кроссплатформенного приложения, использующего для реконструкции объекта алгоритм раскраски вокселей по методу множества гипотез~\cite{multi_hypothesis} и способного задействовать все возможности современного ПК. Из поставленных условий вытекает еще одна задача -- распараллеливание существующего алгоритма.

\subsection{Математические методы}
В качестве базового, для построения модели объекта, используется алгоритм раскраски вокселей по методу множества гипотез~\cite{multi_hypothesis}.

\subsection{Обзор существующих методов решения}

На текущий момент была найдена только одна, находящаяся в открытом доступе, реализация алгоритма раскраски вокселей -- Voxel Coloring Framework~\cite{voxel_coloring_framework}. К её достоинствам можно отнести распространение под академической свободной лицензией~\cite{academic_free_license}. Однако данная реализация обладает следующими недостатками:
\begin{itemize}
\item Работает только в операционной системе семейства MS Windows
\item Для работы необходим пакет Matlab
\item Не использует вычислительные возможности современных графических адаптеров
\end{itemize}
