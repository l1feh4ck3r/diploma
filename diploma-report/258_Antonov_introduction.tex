\section{Введение}

\subsection{Глоссарий}
\textbf{Биндинг} -- программный код, который является прослойкой между какой-либо библиотекой (как правило, написанной на низкоуровневом языке вроде Си) и другим программным кодом, написанным на другом языке программирования (как правило, высокоуровневом, например, Tcl). Таким образом, биндинги <<связывают>> (англ. <<to bind>> — связывать, привязывать) <<чужеродный>> код с <<нашим>>, являясь своего рода прослойкой.~\cite{bindings}

\textbf{Воксел} -- элемент объёмного изображения, содержащий значение элемента растра в трёхмерном пространстве. Вокселы являются аналогами пикселов для трехмёрного пространства.~\cite{wiki_voxel}

\textbf{Кроссплатформенное программное обеспечение} -- программное обеспечение, работающее более чем на одной аппаратной платформе и/или операционной системе.~\cite{wiki_crossplatfom}

\textbf{ПК} -- персональный компьютер.

\textbf{Реконструкция} -- построение объемной модели реального объекта.

\textbf{Фреймворк} -- в информационных системах структура программной системы; программное обеспечение, облегчающее разработку и объединение разных компонентов большого программного проекта. В его состав могут входить вспомогательные программы, библиотеки кода, язык сценариев и проч.~\cite{wiki_framework}

\subsection{Описание предметной области}
Потребность в реалистичном отображении окружающего мира увеличивает значимость трехмерного (3D) моделирования -- построения компьютерных моделей объектов. 3D модели облегчают планирование, контроль и принятие решений во многих отраслях. Например, если в ходе эксплуатации модели требуется внести коррективы, компьютерный объект позволит это сделать максимально быстро.

Особенно сложной проблемой является создание точных моделей объектов из реального мира. Как быстро выяснилось, человеческий глаз очень легко определяет погрешности в синтезированных изображениях хорошо известных ему в повседневной жизни объектов.

Вот несколько примеров объектов, требующих реконструкции:
\begin{enumerate}
\item Здания и строения
\item Предприятия со сложной структурой (нефтегазоперерабатывающие комплексы, химические предприятия и т.д.)
\item Дороги и дорожные объекты (мосты, путепроводы, прилегающая зона)
\item Открытые и закрытые горные разработки
\item Рельефы местности
\end{enumerate}
