\renewcommand{\refname}{Список литературы}
\begin{thebibliography}{99}

	\bibitem{academic_free_license}
		Academic Free License ("AFL") v. 3.0 [Электронный~ресурс] : электрон.~энциклопедия -- Режим доступа:
		\url{http://www.opensource.org/licenses/academic.php}

	\bibitem{opencl_ati_support}
		ATI, ATI Stream Software Development Kit [Электронный~ресурс]
		 -- Режим доступа:
		\url{http://developer.amd.com/gpu/ATIStreamSDK/Pages/default.aspx#two}
	
	\bibitem{bsd}
		<<Simplified BSD License>> [Электронный~ресурс] : электрон.~энциклопедия -- Режим доступа:
		\url{http://www.opensource.org/licenses/bsd-license.php}
	
	\bibitem{chien}
		Chien C.H., Aggarwal J.K. Identification of 3D Objects from Multiple Silhouettes Using Quadtrees/Octrees // 
		Computer Vision Graphics And Image Processing -- 1986 -- PP. 256-273

	\bibitem{opencl_bindings}
		Experimental C++ Bindings to OpenCL [Электронный~ресурс] -- Режим доступа:
		\url{http://www.khronos.org/registry/cl/}
	
	\bibitem{foxc}
		Fixstars OpenCL Cross Compiler [Электронный~ресурс] -- Режим доступа:
		\url{http://www.fixstars.com/en/foxc/}
		
	\bibitem{opencl_ibm_support}
		IBM, OpenCL Development Kit for Linux on Power [Электронный~ресурс]
		-- Режим доступа:
		\url{http://www.alphaworks.ibm.com/tech/opencl}

	\bibitem{laser_scanner}
		Leica ScanStation 2 3D Laser Scanner [Электронный~ресурс] //
		FLT~Geosystems~website -- Режим доступа:
		\url{http://www.fltgeosystems.com/}

	\bibitem{multi_hypothesis}
		Eisert, P. Multi-Hypothesis, volumetric reconstruction of 3-d objects from multiple calibrated camera views /
		Peter Eisert, Eckehard Steinbach, Bernd Girod //
		ICASSP’99, Phoenix, USA -- march~1999. -- PP. 3509-3512

	\bibitem{opencl_standart}
		Khronos Group, OpenCL [Электронный~ресурс] -- Режим доступа:
		\url{http://www.khronos.org/opencl/}

	\bibitem{voxel_coloring_framework}
		Koen van de Sande, Voxel Coloring Framework [Электронный~ресурс]
		-- Режим~доступа:
		\url{http://voxelcoloring.sourceforge.net/}

	\bibitem{aggarwal}
		Martin W.N., Aggarwal J.K. Volumetric description of objects from multiple views //
		IEEE Transactions on Pattern Analysis and Machine Intelligence. -- 1983

	\bibitem{qt_qrectf}
		Nokia, QRectF [Электронный~ресурс]
		-- Режим доступа:
		\url{http://doc.trolltech.com/4.6/qrectf.html}

	\bibitem{qt_qstring}
		Nokia, QString [Электронный~ресурс]
		-- Режим доступа:
		\url{http://doc.trolltech.com/4.6/qstring.html}

	\bibitem{opencl_nvidia_support}
		NVIDIA, CUDA GPUs [Электронный~ресурс] -- Режим доступа:\\
		\url{http://www.nvidia.com/object/cuda_gpus.html}

	\bibitem{nvidia_gpu_sdk}
		NVIDIA GPU Computing SDK code samples [Электронный~ресурс] -- Режим доступа:\\
		\url{http://developer.nvidia.com/object/cuda_3_0GPU Computing SDK code samples_downloads.html}

	\bibitem{potmesil}
		Potmesil M. Generating Octree Models of 3D Objects from Their Silhouettes in a Sequence of Images // 
		Computer Vision Graphics And Image Processing -- 1987 -- PP. 1-29

	\bibitem{opencl_s3_support}
		S3 Graphics [Электронный~ресурс] -- Режим доступа:
		\url{http://www.s3graphics.com/}

	\bibitem{szeliski}
		Szeliski R. Rapid octree construction from image sequences // 
		Computer Vision, Graphics and Image Processing -- 1993 -- PP. 23–32
	
	\bibitem{bindings}
		Tkabber wiki, раздел <<Терминология>> [Электронный~ресурс] -- Режим доступа:
		\url{http://ru.tkabber.jabe.ru/}
	
	\bibitem{wiki_voxel}
		Воксел [Электронный~ресурс] : электрон.~энциклопедия -- Режим доступа: \url{http://ru.wikipedia.org/w/index.php?title=Voxel&oldid=22172795}

	\bibitem{wiki_crossplatfom}
		Кроссплатформенное программное обеспечение [Электронный~ресурс] : электрон.~энциклопедия -- Режим доступа:
		\url{http://wikipedia.org/}

	\bibitem{komarova_voxel_coloring}
		Комарова, Н. Практическая реализация алгоритма раскраски вокселей : курсовая работа / 
		Н.~Комарова~--~2005.~-~23 с.

	\bibitem{wiki_framework}
		Фреймворк [Электронный~ресурс] : электрон.~энциклопедия -- Режим доступа:
		\url{http://ru.wikipedia.org/wiki/Framework}

\end{thebibliography}

