\section{Проект}
\subsection{Средства реализации}
Для реализации приложения был выбран язык программирования C++, т.к. он очень распространен, что упрощает последующую поддержку приложения. К тому же, программы, написанные на этом языке, могут быть скомпилированы для очень большого количества платформ.

Для упрощения переноса приложения на другие платформы, был использован кроссплатформенный инструментарий разработки ПО под названием Qt. Он так же включает в себя инструментарий, необходимый для разработки кроссплатформенных приложений с графическим интерфейсом пользователя. Выбор этого инструментария позволяет уменьшить усилия, требуемые для реализации новой и поддержки имеющейся функциональности.

Для того, чтобы приложение могло задействовать как можно больше доступных ресурсов современных ПК, было решено использовать фреймворк OpenCL. Это позволит приложению задействовать ресурсы всех процессоров в многопроцессорных системах, а так же ресурсы современных видеокарт.
Для взаимодействия с OpenCL используется библиотека OpenCLxx~\cite{opencl_openclxx}, которая предоставляет ООП-интерфейс для доступа к OpenCL.

\subsection{Модули и алгоритмы}

\subsection{Проект интерфейса}

